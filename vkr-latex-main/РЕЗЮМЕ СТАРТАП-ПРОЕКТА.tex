\newsection
\centertocsection{РЕЗЮМЕ СТАРТАП-ПРОЕКТА}
Название: «Бизнес-проект «Программно-информационная система для конструирования чат-ботов для Telegram»».
Проект представляет собой чат-бота в Telegram, предназначенного для автоматизации различных задач, таких как прогноз погоды, модерация контента, ответы на часто задаваемые вопросы и другие.
Были проанализированы современные аналоги и выявлены 
\emph{основные проблемы} , с которыми сталкиваются пользователи медиа-платформ:

1.	Сложность настройки и использования: Многие платформы требуют значительных усилий для настройки и конфигурации ботов, что делает их трудными для новичков.

2.	Ограниченные возможности кастомизации: Существующие решения часто предлагают ограниченные возможности для настройки и кастомизации ботов под специфические потребности.

3.	Высокая стоимость: Множество платформ требуют значительных финансовых затрат для доступа к продвинутым функциям и интеграциям.

4.	Ограниченная интеграция с внешними сервисами: Поддержка интеграции с внешними сервисами и API может быть ограниченной или сложной в настройке.

5.	Недостаточная поддержка и документация: Пользователи часто сталкиваются с недостаточной поддержкой и неполной документацией, что усложняет решение возникающих проблем.

Многофункциональный чат-бот призван решить эти проблемы и стать удобным пользователям.

\emph{Актуальность} данного чат-бота связана с высоким спросом на автоматизированные решения, которые могут помочь в управлении и взаимодействии с участниками.
\emph{Цель}. Разработка чат-бота для Telegram, который будет помогать пользователям решать однотипные задачи, автоматизировать рутинные процессы и улучшать взаимодействие внутри сообществ.

\emph{Отличительные особенности} приложения по сравнению с существую- щими решениями и технологиями:

1.	Автоматизация задач. Чат-бот позволяет автоматизировать рутинные задачи, такие как ответы на часто задаваемые вопросы, модерация контента и управление задачами.

2.	Централизованная обратная связь. Возможность получения и обработки обратной связи от пользователей в одном месте.

3.	Эффективное управление информацией. Бот помогает структурировать и организовывать информацию, что облегчает её поиск и обработку.

В разработанном приложении реализованы следующие возможности: автоматические ответы на часто задаваемые вопросы, модерация и фильтрация контента, получение данных о погоде в своем городе.

При разработке чат-бота были решены следующие задачи: создание модулей для автоматических ответов на часто задаваемые вопросы, разработка системы модерации и фильтрации контента, реализация возможностей интеграции с популярными внешними сервисами.

\emph{Идея} создания проекта возникла в сентябре 2023 года, тогда же и началась разработка. С февраля 2024 года началось тестирование чат-бота  путём предоставления доступа онлайн пользователям, с июля 2024 года планируется расширение аудитории тестирования путём развёртывания приложения на общедоступном сервере для тестирования работы приложения под нагрузкой, на этом этапе принимать участие в тестировании смогут пользователи по всей стране. 

Для защиты исключительных прав и правовой охраны будет осуществлена регистрация программного средства.
Самыми значительными и высоко-вероятными рисками для проекта
являются рост издержек и инфляция.

\emph{Перспективой} дальнейшей разработки программной системы являет- ся наращивание аудитории, увеличение количества созданных ботов и расширение функционала посредством обновлений.

\emph{Бизнес-цели}:

1.	Увеличить число пользователей, которые воспользовались ботом: 1000 пользователей через год. Для достижения этой цели можно использовать различные методы привлечения пользователей, например, рекламные кампании в социальных сетях, поисковую оптимизацию и т.д.

2.	Добавить не менее 5 новых ключевых функций в течение первого года на основе пользовательских отзывов и анализа рынка. Регулярно собирать обратную связь от пользователей, анализировать данные о взаимодействии с ботом, следить за новыми тенденциями и технологиями.

3.	Достигнуть уровня удовлетворенности пользователей не менее 85 процентов по результатам регулярных опросов. Постоянно улучшать качество чат-бота, оперативно реагировать на жалобы и претензии, обеспечивать высокий уровень поддержки пользователей

