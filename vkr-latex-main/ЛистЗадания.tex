\newsection
\begin{center}
\large\textbf{Минобрнауки России}

\large\textbf{Юго-Западный государственный университет}
\vskip 1em
\normalsize{Кафедра программной инженерии}
\vskip 1em

\begin{flushright}
\begin{tabular}{p{.4\textwidth}}
\centrow УТВЕРЖДАЮ: \\
\centrow Заведующий кафедрой \\
\hrulefill \\
\setarstrut{\footnotesize}
\centrow\footnotesize{(подпись, инициалы, фамилия)}\\
\restorearstrut
«\underline{\hspace{1cm}}»
\underline{\hspace{3cm}}
20\underline{\hspace{1cm}} г.\\
\end{tabular}
\end{flushright}
\end{center}
\section*{ЗАДАНИЕ НА ВЫПУСКНУЮ КВАЛИФИКАЦИОННУЮ РАБОТУ
  ПО ПРОГРАММЕ БАКАЛАВРИАТА}
{\parindent0pt
  Студента \АвторРод, шифр\ \Шифр, группа \Группа
  
1. Тема «\Тема\ \ТемаВтораяСтрока» утверждена приказом ректора ЮЗГУ от \ДатаПриказа\ № \НомерПриказа.

2. Срок предоставления работы к защите \СрокПредоставления

3.	Общая часть ВКР:

4.	Индивидуальная часть ВКР:
  
  4.1. Исходные данные:
  
 	 4.1.1.	Перечень решаемых задач
 		
 		1)	разработать модель данных программной системы; определить варианты использования программной системы; разработать требования к программной системе;
 	
 		2)	осуществить проектирование программной системы; разработать архитектуру программной системы; разработать пользовательский интерфейс для создания и управления чат-ботом;
  	
  		3) зарегистрировать бота в Telegram с помощью @BotFather и получить токен для доступа к API
  	
  		4)	провести тестирование модуля межпользовательского взаимодействия программной системы; провести системное тестирование компонентов программной системы.
  
  	4.1.2.	Входные данные и требуемые результаты для программы:
  	1)	Входными данными для программной системы являются: команды пользователей; сообщения пользователей; данные для интеграции с внешними сервисами.
  	2)	Выходными данными для программной системы являются: ответы на команды; ответы на текстовые сообщения; модерация.
  4.2.	Содержание работы (по разделам):
 	4.2.1.	Введение
 	4.2.2.	Резюме стартап-проекта
	4.2.3.	Анализ предметной области
  

\begin{enumerate}[label=Лист \arabic*.]
\item Сведения о ВКРБ
\item Цели и задачи разработки
\item Диаграммы вариантов использования
\item Диаграмма развертывания приложения
\item Заключение
\end{enumerate}

\vskip 2em
\begin{tabular}{@{}p{6.8cm}C{3.8cm}C{4.8cm}}
Руководитель ВКР & \lhrulefill{\fill} & \fillcenter\Руководитель\\
\setarstrut{\footnotesize}
& \footnotesize{(подпись, дата)} & \footnotesize{(инициалы, фамилия)}\\
\restorearstrut
Задание принял к исполнению & \lhrulefill{\fill} & \fillcenter\Автор\\
\setarstrut{\footnotesize}
& \footnotesize{(подпись, дата)} & \footnotesize{(инициалы, фамилия)}\\
\restorearstrut
\end{tabular}
}
