\newsection
\section{Техническое задание}
\subsection{Основание для разработки}

Функциональное назначение разрабатываемой программной системы заключается в предоставлении.
Предполагается, что разрабатываемая программная система будет ис- пользоваться широким кругом лиц, занимающихся цифровым искусством в России.

\subsection{Назначение разработки}

Функциональное назначение разрабатываемой программной системы заключается в автоматизации и оптимизации однотипных задач пользователей в группах и каналах Telegram. команды пользователей; сообщения пользователей; данные для интеграции с внешними сервисами[9].
\subsection{Требования к программной системе}
\subsubsection{Требования к программной системе}

Входными данными для программной системы являются:

-	id чата;

-	id пользователя;

-	команды пол;

-	приглашение в чат;

-	запрос 

-	запрос на создание ключевого слова;

-	запрос на удаление ключевого слова;

-	запрос на редактирование ключевого слова;

-	запрос на список ключевых слов;

-	запрос на создание ключевого слова с уведомлением;

-	запрос на удаление ключевого слова с уведомлением ;

-	запрос на редактирование ключевого слова с уведомлением ;

-	запрос на список ключевых слов с уведомлением ;

Выходными данными для программной системы являются:

-	сообщения о успешной обработке команд;

-	отправка сообщений по ключевому слову;

-	отправка сообщений по ключевому слову с уведомлением;

-	список ключевых слов с уведомлением;

-	отправка информации о погоде.



\subsubsection{Функциональные требования к программной системе}

В разрабатываемой программной системе для пользователя должны быть реализованы следующие функции:

1.	Старт и помощь.

2.	Сохранение сообщения или фото.

3.	Сохранение сообщения или фото с уведомлением.

4.	Редактирование сообщения или фото.

5.	Редактирование сообщения или фото с уведомлением.

6.	Удаление сообщения или фото.

7.	Удаление сообщения или фото с уведомлением.

8.	Просмотр списка ключевых слов.

9.	Просмотр списка ключевых слов с уведомлением.

10.	Фильтрация запрещенных слов.

11.	Ответ на ключевое слово.

12.	Просмотр прогноза погоды.

13.	Выбор города для просмотра прогноза погоды.

14.	Изменение города для получения прогноза погоды.

На рисунках 2.1 представлены функциональные требования к системе в виде диаграммы прецедентов нотации UML.


\begin{figure}
	\center{\includegraphics[width=1\linewidth]{un3}}
	\caption{Диаграмма вариантов использования}
	\label{un3:image}
\end{figure}
\vspace{-\figureaboveskip} % двойной отступ не нужен (можно использовать, если раздел заканчивается картинкой)

\paragraph{Вариант использования «Попытка написать сообщение содержащие нецензурную лексику»}

Заинтересованные лица и их требования: пользователь хочет написать сообщение содержащие нецензурную лексику в чате сообщества.

Предусловие: пользователь использует чат в котором присутствует чат бот.

Постусловие: пользователь написал сообщение содержащие нецензурную лексику.
 
Основной успешный сценарий:

1.	Пользователь отправляет сообщение содержащие нецензурную лексику.

2.	Сообщение удаляется мгновенно.




\paragraph{Вариант использования «Получение ежедневного прогноза погоды в чате»}

Заинтересованные лица и их требования: пользователь хочет настроить получение ежедневного прогноза погоды в чате города.

Предусловие: пользователь использует чат в котором присутствует чат-бот.

Постусловие: пользователь хочет получать прогноз погоды по одному существующему городу.
Основной успешный сценарий:

1.	Пользователь переходит на в чат, котором добавлен бот.

2.	Пишет команду «/set\_city».

3.	Бот отправляет сообщение о успешном срабатывании команды и ожидании информации от пользователя с названием города для предоставления прогноза.

4.	Пользователь отправляет название города.

5.	Бот сохраняет информацию по данному чату и каждый день в 9:00 отправляет прогноз погоды на день.
\paragraph{Вариант использования «Получение прогноза погоды в чате в данный момент»}

Заинтересованные лица и их требования: пользователь хочет получить прогноз погоды сейчас.

Предусловие: пользователь использует чат в котором присутствует чат-бот.

Постусловие: сохранена информация о городе.

Основной успешный сценарий:

1.	Пользователь пишет команду «/current\_weather».

2.	Бот отправляет прогноз погоды.

\paragraph{Вариант использования «Попытка написать сообщение содержащие нецензурную лексику»}

Заинтересованные лица и их требования: пользователь хочет написать сообщение содержащие нецензурную лексику в чате сообщества.

Предусловие: пользователь использует чат в котором присутствует чат бот.

Постусловие: пользователь написал сообщение содержащие нецензурную лексику. 

Основной успешный сценарий:

1.	Пользователь отправляет сообщение содержащие нецензурную лексику.

2.	Сообщение удаляется мгновенно.
\subsubsection{Нефункциональные требования к программной системе}
\paragraph{Требования к надежности}
Система должна работать все время, за исключением плановых технических обслуживаний.
\paragraph{Требования к безопасности}
Все пользователи должны быть зарегистрированы в Telegram. Доступ к различным функциям чат-бота должен быть основан на ролях и разрешениях в группе или канале.
\paragraph{Требования к программному обеспечению}
Для реализации программной системы должны быть использованы следующие языки программирования:

Python 3.7 или выше: бот написан на языке программирования Python, поэтому необходимо наличие установленного интерпретатора Python версии 3.7 или выше.

Операционная система: поддерживаются любые ОС, совместимые с Python, такие как Linux, Windows, macOS.
\subsection{Требования к оформлению документации}
Требования к стадиям разработки программ и программной докумен- тации для вычислительных машин, комплексов и систем независимо от их назначения и области применения, этапам и содержанию работ устанавлива- ются ГОСТ 19.102-77.

Программная документация должна включать в себя:

1.	Техническое задание.

2.	Технический проект.

3.	Рабочий проект.

