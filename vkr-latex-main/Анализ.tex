\newsection
\section{Анализ предметной области}

С ростом популярности мессенджеров, таких как Telegram, все больше команд и сообществ используют их для координации и общения. Это привело к необходимости автоматизации рутинных и однотипных задач, что позволяет снизить нагрузку на администраторов и повысить общую эффективность взаимодействия. Telegram-боты предоставляют отличную платформу для реализации этих целей благодаря их гибкости и широким возможностям интеграции.

Автоматизация однотипных задач включает в себя выполнение рутинных, повторяющихся действий без участия человека. В контексте Telegram-ботов это может включать:

•	Ответы на часто задаваемые вопросы (FAQ): Бот могут автоматически отвечать на распространенные вопросы, экономя время администраторов.

•	Прогноз погоды: Бот может определять прогноз погоды в любой точке мира.

•	Модерация: Бот могут автоматически фильтровать и удалять нежелательный контент, предупреждать или блокировать нарушителей.

•	Интеграция с внешними сервисами: Получение и обработка данных из различных источников для выполнения задач.

В ходе анализа целевой аудитории были составлены обобщающие мо- дели персонажей.

На рисунке 1.1 изображена модель персонажа, польуюегося чатом.

\begin{figure}
	\center{\includegraphics[width=1\linewidth]{un2}}
	\caption{Модель персонажа, администратора Telegram канала}
	\label{un2:image}
\end{figure}

Ниже описана Journey Map пользователя по использованию для публика- ции цифрового искусства.

1.	Определение потребности. Дмитрий, администратор популярного Telegram-канала, сталкивается с несколькими проблемами. Ему необходимо предоставить участникам точный прогноз погоды, автоматически удалять нецензурные сообщения и ссылки, а также отвечать на часто задаваемые вопросы. Осознавая, что выполнение этих задач вручную занимает много времени и сил, Дмитрий решает найти чат-бота, который сможет автоматизировать эти процессы.

2.	Поиск вариантов. Дмитрий начинает искать варианты ботов для модерирования своего канала. Он может использовать поисковые системы, спросить у своих коллег, где они ищут модерацию для своих каналов или найти информацию в социальных сетях.

3.	Открытие существующих сайтов. Дмитрий переходит на несколько сайтов для поиска нужного чат-бота, чтобы ознакомиться с возможностями каждого из них.
 
4.	Окончательный выбор. После ознакомления с несколькими сайтами, Дмитрий принимает окончательное решение, какого чат-бота ему добавить к себе в канал. Он может выбрать чат-бота, который соответствует его целям и потребностям как администратора. После этого он добавляет бота к себе на канал и пользуется.

На рисунке 1.2 изображена модель персонажа, интересующегося изысканной готовкой.

\begin{figure}
	\center{\includegraphics[width=1\linewidth]{un1}}
	\caption{Модель персонажа, интересующегося изысканной готовкой}
	\label{uml1:image}
\end{figure}


1.	Определение потребности. Анна, управляющая популярной кулинарной группой в Telegram, заметила, что участники часто задают одни и те же вопросы о рецептах, ингредиентах и технике приготовления. Кроме того, ей нужно поддерживать чистоту в группе, удаляя рекламные ссылки и неуместные сообщения. Анна понимает, что ей нужен чат-бот, который сможет автоматизировать ответы на часто задаваемые вопросы, предоставлять рецепты по запросу и модерировать контент группы[5].

2.	Поиск вариантов. Анна начинает исследовать возможные варианты чат-ботов для своей группы. Она читает статьи и обзоры на специализированных сайтах, посещает форумы, посвященные кулинарным сообществам, и задает вопросы коллегам-администраторам других групп. Она также просматривает отзывы пользователей на различных платформах, чтобы понять, какие боты хорошо справляются с поставленными задачами.

3.	Открытие существующих сайтов. Анна посещает несколько сайтов, предлагающих Telegram-ботов, и тщательно изучает их функционал. 

4.	Окончательный выбор. После ознакомления с несколькими сайтами, Анна принимает окончательное решение, каким ботом она будет пользоваться в своем канале. Она добавляет его к себе в группу и настраивает его под свои нужды. Бот начинает автоматически отвечать на вопросы участников, предоставлять рецепты и удалять неуместный контент. Благодаря этому Анна освобождает значительное количество времени, которое она теперь может посвятить созданию нового контента и взаимодействию с участниками группы.

В ходе анализа предметной области были изучены конкурентные решения. Выбор конкретного бота зависит от конкретных потребностей и приоритетов администратора канала или группы. Данный чат-бот может занять свою нишу, предлагая уникальное сочетание функциональности, простоты использования и адаптируемости под конкретные задачи пользователей.

Были выделены несколько конкурентов, которые предоставляют сход- ную функциональность. Они представлены ниже:

1.	ManyBot.

2.	Chatfuel.

3.	Dialogflow.

Одним из основных конкурентов данного чат-бота является конструктор ManyBot, который позволяет создавать ботов без навыков программирования[6]. Он предоставляет функции для автоматических ответов на часто задаваемые вопросы, отправки уведомлений, создания опросов и управления контентом. Недостатки ManyBot описаны ниже:

1.	Ограниченный набор команд, бот предлагает базовый набор команд и функций, что может не удовлетворять потребности пользователей, которым требуется более продвинутый функционал.

2.	Отсутствие интеграции с внешними сервисами: Для пользователей, которым нужна интеграция с конкретными сервисами (например, метеорологическими API для прогноза погоды), ManyBot может быть недостаточно гибким.

3.	Отсутствие автоматической модерации: ManyBot не предоставляет продвинутых инструментов для автоматической модерации, таких как анализ контента с помощью AI для выявления спама или оскорбительного контента.

Следующим конкурентом является Chatfuel – это мощная платформа для создания ботов, которая поддерживает интеграцию с Telegram. Chatfuel предоставляет множество функций, включая автоматические ответы, напоминания, опросы и модерацию контента. Платформа также предлагает интеграцию с различными внешними сервисами и API[7]. Недостатки Chatfuel описаны ниже:

1.	Несмотря на наличие графического интерфейса, Chatfuel может быть сложным для новичков. Понимание логики построения бота и настройки всех необходимых блоков требует времени и усилий.

2.	Хотя Chatfuel позволяет использовать JSON API, возможности для продвинутой кастомизации и программирования могут быть ограничены по сравнению с другими платформами, которые предоставляют полный доступ к исходному коду ботов.

Другим конкурентом является Dialogflow от Google позволяет создавать интеллектуальных чат-ботов с использованием машинного обучения и обработки естественного языка. Он поддерживает интеграцию с Telegram и предоставляет инструменты для создания сложных сценариев взаимодействия, автоматических ответов и модерации контента. Недостатки Dialogflow описаны ниже:

1.	Использование Dialogflow часто требует интеграции с другими сервисами Google Cloud, что может привести к высоким затратам, особенно для крупных проектов с высокой нагрузкой[8]. 

2.	Интерфейс Dialogflow может быть менее интуитивным по сравнению с другими платформами. Некоторые пользователи отмечают, что настройка агентов и управление ими может быть сложной задачей из-за недостаточно продуманного интерфейса. 

3.	Процесс тестирования и отладки ботов в Dialogflow может быть сложным и требовать значительных усилий. Найти и исправить ошибки в сложных сценариях взаимодействия может быть непросто.
 
4.	Неожиданные расходы: При активном использовании платформы затраты могут быстро расти, особенно если бот обрабатывает большое количество запросов.


 
