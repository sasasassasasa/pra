\newsection
\centertocsection{СПИСОК ИСПОЛЬЗОВАННЫХ ИСТОЧНИКОВ}

%\begin{hyphenrules}{nohyphenation} %отключение переноса слов в содержании
\begin{thebibliography}{9}

\bibitem{1} Макконнелл , С. Совершенный код. Практическое руководство по разработке программного обеспечения / С. Макконнелл . – СПб. : БХВ, 2022. – 896 с. – ISBN 978-5-9909805-1-8. – Текст : непосредственный.

\bibitem{2} Марков А.А., "Чат-боты. Что такое и зачем нужны". Статья из журнала "Хакер", №8 (2017), с. 64-69

\bibitem{3}	Херманс , Ф. Ум программиста. Как понять и осмыслить любой код. / Ф. Херманс . – СПб. : БХВ, 2023. – 272 с. – ISBN 978-5-9775-1176-6. – Текст :
непосредственный.


\bibitem{4} Брайант, Э. Компьютерные системы. Архитектура и программирование / Э. Брайант, О’Халларон Р. – Москва : ДМК Пресс, 2022. – 994 с. – ISBN 978-5-97060-492-2. – Текст : непосредственный.

\bibitem{5}	Иванов И.П., "Разработка и обучение чат-бота для Telegram на PHP". Статья из журнала "Web-программист", №4 (2018), с. 20-25. – Текст : непосредственный.

\bibitem{6}	Соколов А.К., "Использование Telegram API для создания чат-ботов". Книга. Москва: Техносфера, 2019. 240 с. ISBN 978-5-94836-777-2. – Текст : непосредственный.

\bibitem{7}	Мартин, Р. С. Чистый код. Создание, анализ и рефакторинг / Р. С. Мартин. – Санкт-Петербург : Питер, 2021. – 464 с. – ISBN 978-5-4461-0960– Текст~: непосредственный.

\bibitem{8}	Белов Д.М., "Применение машинного обучения для улучшения функционала чат-ботов". Статья из журнала "Искусственный интеллект", №2 (2020), с. 45-52. - Текст~: непосредственный.

\bibitem{9}	Петцольд, Ч. Код. Тайный язык информатики / Ч. Петцольд. – Москва
: Манн, Иванов и Фербер, 2019. – 448 c. – ISBN 978-5-00117-545-2. – Текст : непосредственный.


\bibitem{10} Мартин, Р. С. Чистая архитектура. Искусство разработки программного обеспечения / Р. С. Мартин. – Санкт-Петербург : Питер, 2018. – 351 c. – ISBN 978-5-4461-0772-8.– Текст : непосредственный

\bibitem{11} Петров Н.С., "Особенности использования API Telegram при разработке чат-ботов". Материалы конференции "Современные информационные технологии", 2019, с. 112-118 – Текст~: непосредственный.

\bibitem{12} Джувел, Л. Совершенный софт / Л. Джувел. – Санкт-Петербург : Питер, 2020. – 480 с. – ISBN 978-5-4461-1621-8. – Текст : непосредственный.

\bibitem{13} 	Фленов, М.Е. Библия C\#. 5-е издание. / М.Е. Фленов. – СПб. : БХВ, 2022. – 464 с. – ISBN 978-5-9775-6827-2. – Текст : непосредственный.

\bibitem{14}	 Троелсен, Э. Язык программирования C\# 9 и платформа .NET 5: основные принципы и практики программирования, 10-е издание. / Э. Троелсен, Ф. Джепикс. – Москва : Диалектика, 2022. – 1392 с. – ISBN 978-5- 907458-67-3. – Текст : непосредственный.

\bibitem{15}  Джепикс, Ф. Язык программирования C\# 7 и платформы .NET и
.NET Core / Ф. Джепикс, Э. Троелсен. – Москва : Вильямс, 2018. – 1328 c. – ISBN: 978-1-4842-3017-6. – Текст : непосредственный.


\bibitem{16} Дронов, В.А. JavaScript. 20 уроков для начинающих. / В.А. Дронов. – СПб. : БХВ, 2020. – 352 с. – ISBN 978-5-9775-6589-9. – Текст :
непосредственный.

\bibitem{17} Дронов , В.А. JavaScript. Дополнительные уроки для начинающих. / В.А. Дронов . – СПб. : БХВ, 2021. – 352 с. – ISBN 978-5-9775-6781-7. – Текст :
непосредственный.

\bibitem{18} Фримен, А. Практикум по программированию на JavaScript / А. Фримен. – Москва : Вильямс, 2013. – 960 с. – ISBN 978-5-8459-1799-7. –
Текст : непосредственный.

\bibitem{19} Умрихин, Е.Д. Разработка веб-приложений с помощью ASP.Net Core MVC. / Е.Д. Умрихин. – СПб. : БХВ, 2023. – 416 с. – ISBN 978-5-9775-
1206-0. – Текст : непосредственный.


\bibitem{20} Вишняков Ю.М., Кодачигов В.И., Родзин С.И. Учебно–методическое пособие по курсам «Системы искусственного интеллекта», «Методы распознавания образов». Таганрог: Из–во ТРТУ, 2021.~- 30 с.~– ISBN: 978-8521479630.~– Текст~: непосредственный.

\bibitem{21}	Ганди, Р. Head First. Git / Р. Ганди. – СПб. : БХВ, 2023. – 464 с. – ISBN 978-5-9775-1777-5. – Текст : непосредственный.

\bibitem{22} Чакон, C. Git для профессионального программиста / C. Чакон. – Санкт-Петербург : Питер, 2016. – 496 с. – ISBN 978-5-496-01763-3. – Текст : непосредственный.

\bibitem{23} Хориков, В. Принципы юнит-тестирования / В. Хориков. – СПб. : Питер, 2022. – 320 с. – ISBN 978-5-4461-1683-6. – Текст : непосредственный.

\bibitem{24} Гома X. UML. Проектирование систем реального времени, параллельных и распределенных приложений: Пер. с англ. М.: ДМК Пресс, 2022.~- 87 с.~– ISBN: 978-1234567894.~– Текст~: непосредственный.

\bibitem{25} Маурисио, А. Эффективное тестирование программного обеспечения / А. Маурисио. – Москва : ДМК Пресс, 2023. – 370 с. – ISBN 978-5-97060-997-2. – Текст : непосредственный.

\bibitem{26} Гультяев А.К. MATLAB 5.3. Имитационное моделирование в среде Windows, М.: Корона принт, 2021.~- 33 с.~– ISBN: 978-1234567896.~– Текст~: непосредственный.

\bibitem{27} Игнатьев, А. В. Тестирование программного обеспечения / А. В. Игнатьев. – Москва : Лань, 2021. – 56 с. – ISBN 978-5-8114-8072-2. – Текст : непосредственный.

\bibitem{28} Плаксин, М. А. Тестирование и отладка программ для профессионалов будущих и настоящих / М. А. Плаксин. – Москва : БИНОМ, 2020. – 170 с. – ISBN 978-5-00101-810-0. – Текст: непосредственный.

\bibitem{29} Дьяконов В. Mathematica 4: учебный курс. СПб: Питер, 2022.~- 162 с.~– ISBN: 978-1234567899.~– Текст~: непосредственный.

\bibitem{30} Шмитт, Э. Применение Web-стандартов. CSS и Ajax для больших сайтов / Э. Шмитт, К. Блессинг, Р. Черни . – СПб. : Корона-Принт, 2016. – 224 с. – ISBN 978-5-7931-0844-7. – Текст : непосредственный.

\bibitem{31}Вайсфельд,	М.	Объектно-ориентированное	мышление	/	М. Вайсфельд. – СПб : Питер, 2014. – 304 с. – ISBN 978-5-496-00793-1. – Текст :
непосредственный.
\bibitem{32} Джонсон, Р. Приемы объектно-ориентированного проектирования. Паттерны проектирования / Р. Джонсон, Г. Эрих, Р. Хелм, Д. Влисседес. – Санкт-Петербург : Питер, 2016. – 366 с. – ISBN 978-5-459-01720-5. – Текст : непосредственный.

\bibitem{33} Вайсфельд,	М.	Объектно-ориентированное	мышление	/	М. Вайсфельд. – СПб : Питер, 2014. – 304 с. – ISBN 978-5-496-00793-1. – Текст :
непосредственный.


\bibitem{34} Гаско, Р. Объектно Ориентированное Программирование / Р. Гаско. – Москва : Солон-Пресс, 2021. – 298 с. – ISBN 978-5-91359-285-9. – Текст :
непосредственный.


\bibitem{35} Ларман, К. Применение UML и шаблонов проектирования. Введение в объектно-ориентированный анализ, проектирование и итеративную разработку: учебник и практикум для бакалавриата и магистратуры / К. Ларман. – М.: ООО “И.Д. Вильямс”, 2013. – 426 с. – ISBN 978-5-8459-1185-8.. – Текст : непосредственный.

\bibitem{36} Агальцов, В.П. Базы данных. Локальные базы данных: учебник /
В.П. Агальцов. - М.: ИД ФОРУМ, НИЦ ИНФРА-М, 2016. - 352 с. - ISBN
978-5-16-011625-9. - Текст: непосредственный

\bibitem{37} Вайсфельд, М. Объектно-ориентированное мышление / М. Вайс- фельд. - СПб.: Питер, 2014. - 304 с. - ISBN 978-5-496 -00793-1. - Текст: непосредственный.

\end{thebibliography}
%\end{hyphenrules}
