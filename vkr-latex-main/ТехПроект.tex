\newsection
\section{Технический проект}
\subsection{Общая характеристика организации решения задачи}

Полное наименование системы: программно-информационная система для конструирования чат-ботов для Telegram.

Проект представляет собой телеграм-бота, выполняющего функции сохранения сообщений, связанных с ключевыми словами, и отправки уведомлений о погоде. Вся структура проекта организована в виде набора модулей, каждый из которых отвечает за определенные функциональные блоки. Рассмотрим основные модули и их функциональность. 

Модуль управления ботом:

- отвечает за взаимодействие с API Telegram для обработки входящих сообщений от пользователей и отправки ответов.

- реализует логику работы бота, включая обработку команд пользователя и пересылку сообщений другим модулям для дальнейшей обработки.
Модуль сохранения сообщений:

- отвечает за сохранение сообщений, содержащих ключевые слова, в базу данных или файловую систему для последующего анализа.

- обеспечивает функционал поиска и извлечения сохраненных сообщений по заданным ключевым словам.

Модуль анализа текста:

- производит анализ входящих сообщений для выявления ключевых слов или фраз, связанных с интересующими темами.

- может использовать методы обработки естественного языка для улучшения точности выявления ключевых элементов.

Модуль уведомлений о погоде:

- получает данные о погоде из внешних источников (например, открытых API погодных сервисов).

- отправляет уведомления пользователям о текущей или прогнозируемой погоде в заданном местоположении или с использованием указанных параметров.

Модуль управления настройками:

- позволяет пользователям настраивать параметры бота, такие как географическое местоположение для уведомлений о погоде, список ключевых слов для отслеживания и другие параметры.

- обеспечивает интерфейс для изменения настроек и сохранения их в базе данных.



\subsection{Проектирование архитектуры программной системы}
\subsubsection{Компоненты программной системы}
Разрабатываемая система является ботом для мессенджера, который предназначен для обработки команд, фильтрации сообщений и управления ключевыми словами.

Для разработки frontend-модуля необходимо использовать язык высокоуровневый язык программирования общего назначения Python.

Для разработки backend-модуля будет использоваться язык программи- рования высокоуровневый язык программирования общего назначения Python.

Используемая среда разработки – PyCharm Community Edition 2023.2.

В качестве системы управления версиями для командной разработки необходимо использовать Git[21][22].

Диаграмма развёртывания программы представлена на рисунке 3.1.
\begin{figure}
	\center{\includegraphics[width=1\linewidth]{un4}}
	\caption{Диаграмма развертывания}
	\label{un4:image}
\end{figure}


\subsubsection{Язык программирования Python}

Выбор языка программирования Python обусловлен его многочисленными преимуществами и подходящими характеристиками для разработки разнообразных приложений:

• Простота и удобство 

Python предлагает простой и понятный синтаксис, что делает его привлекательным для начинающих разработчиков и обеспечивает высокую продуктивность при написании кода.

• Богатая экосистема

Python имеет огромное сообщество разработчиков и богатую экосистему библиотек и инструментов, которые позволяют быстро решать разнообразные задачи, начиная от веб-разработки и машинного обучения, и заканчивая научными вычислениями и анализом данных.

• Кроссплатформенность и переносимость

Python поддерживает все основные операционные системы и может быть запущен на различных платформах без изменений в коде, что обеспечивает легкость развертывания приложений.

• Широкое применение

Python применяется в различных областях, включая веб-разработку, научные исследования, разработку игр, автоматизацию задач, анализ данных и многое другое, что делает его универсальным инструментом для разработчиков.

• Активное сообщество и поддержка 

Python имеет активное и дружественное сообщество разработчиков, готовых помочь и поддержать новичков, а также обеспечивает доступ к богатой документации, форумам обсуждений и онлайн-курсам для обучения.

В целом, Python является идеальным выбором для разработки различных приложений благодаря своей простоте, гибкости, мощным возможностям и обширной экосистеме.
\subsubsection{OS}
Библиотека os является одной из стандартных библиотек языка программирования Python и предоставляет набор функций для взаимодействия с операционной системой. Эта библиотека позволяет выполнять широкий спектр задач, связанных с файловой системой, процессами, переменными окружения и другими аспектами операционной системы.

Основные особенности библиотеки os
Управление файловой системой:

os предоставляет функции для работы с файлами и директориями, такие как создание, удаление, переименование и перемещение файлов и папок.
Поддержка операций чтения и записи файлов, а также получение информации о файлах и директориях.
Работа с процессами:

Возможность запуска внешних команд и программ с помощью функций os.system и os.exec.
Управление процессами, включая создание, завершение и получение информации о процессах.
Переменные окружения:

Доступ к переменным окружения через функции os.environ, os.getenv и os.putenv.
Возможность установки, изменения и удаления переменных окружения.
Работа с путями:

Подмодуль os.path предоставляет удобные функции для работы с путями файловой системы, такие как объединение, разделение и нормализация путей.
Операции, специфичные для платформы:

os предоставляет функции, специфичные для различных операционных систем, что позволяет писать кроссплатформенный код.
Преимущества библиотеки os
Широкий спектр возможностей:

Библиотека предоставляет богатый набор функций для работы с файловой системой, процессами и переменными окружения.
Кроссплатформенность:

os поддерживает множество операционных систем, включая Windows, macOS и Linux, что позволяет писать кроссплатформенные приложения.
Стандартная библиотека:

os является частью стандартной библиотеки Python, поэтому не требует установки дополнительных пакетов и всегда доступна.
Гибкость:

Библиотека предоставляет функции для выполнения как простых, так и сложных операций, связанных с операционной системой.
Недостатки библиотеки os
Ограниченная безопасность:

Некоторые функции, такие как os.system, могут представлять угрозу безопасности, если входные данные не проверяются должным образом, так как они могут быть использованы для выполнения произвольных команд.
Сложность использования некоторых функций:

Для новичков некоторые функции библиотеки могут показаться сложными и требовать дополнительных знаний о работе операционных систем.
Ограничения специфичные для платформы:

Хотя библиотека поддерживает множество операционных систем, некоторые функции могут вести себя по-разному на разных платформах, что может требовать дополнительных проверок и обработки исключений в коде.

Заключение:

Библиотека os является мощным инструментом для взаимодействия с операционной системой из Python-программ. Она предоставляет широкий набор функций для работы с файловой системой, процессами и переменными окружения, что делает её незаменимой для задач автоматизации и системного программирования. Преимущества библиотеки, такие как кроссплатформенность, гибкость и доступность в стандартной библиотеке Python, значительно перевешивают её недостатки.
\subsubsection{pyTelegramBotAPI}

Библиотека pyTelegramBotAPI (также известная как TeleBot) является одной из наиболее популярных библиотек для разработки Telegram-ботов на языке программирования Python. Она предоставляет простой и удобный интерфейс для взаимодействия с API Telegram, что делает процесс создания ботов относительно лёгким даже для начинающих разработчиков. Давайте рассмотрим основные аспекты этой библиотеки, её преимущества и недостатки.

Основные особенности pyTelegramBotAPI
Простота использования:

pyTelegramBotAPI позволяет быстро и легко начать работу с ботом благодаря интуитивно понятному интерфейсу.
Основные функции библиотеки включают создание и отправку сообщений, обработку обновлений (updates), работу с клавиатурами и кнопками, а также управление чатами и пользователями.
Асинхронная обработка:

Библиотека поддерживает асинхронный режим работы, что позволяет обрабатывать несколько запросов одновременно, улучшая производительность бота.
Поддержка всех типов сообщений:

pyTelegramBotAPI поддерживает работу со всеми типами сообщений, включая текст, фото, видео, документы, стикеры и другие мультимедийные файлы.
Обработка команд и текстовых сообщений:

Библиотека предоставляет удобные декораторы для обработки команд и текстовых сообщений, что упрощает организацию кода и улучшает его читаемость.
Преимущества pyTelegramBotAPI
Легкость освоения:

Библиотека имеет простую и понятную документацию, что позволяет разработчикам быстро разобраться с основными функциями и начать разработку бота.
Широкая поддержка:

pyTelegramBotAPI активно поддерживается сообществом и имеет множество примеров и решений на форумах и в репозиториях, таких как GitHub.
Гибкость:

Библиотека позволяет легко расширять функционал бота за счет поддержки различных типов сообщений и интеграции с другими библиотеками Python.
Асинхронность:

Поддержка асинхронного режима работы позволяет эффективно обрабатывать большое количество запросов и улучшает отзывчивость бота.
Недостатки pyTelegramBotAPI
Производительность:

В некоторых случаях производительность библиотеки может быть ниже по сравнению с другими решениями, особенно если бот обрабатывает большое количество запросов в режиме реального времени.
Зависимость от внешних библиотек:

Для работы в асинхронном режиме библиотека требует установки дополнительных зависимостей, что может усложнить настройку окружения.
Ограниченная гибкость в конфигурации:

Некоторые разработчики могут столкнуться с ограничениями в конфигурации и кастомизации поведения бота, что потребует использования дополнительных библиотек или написания собственного кода.
Заключение
pyTelegramBotAPI является отличным выбором для создания Telegram-ботов благодаря своей простоте, широким возможностям и активной поддержке сообщества. Она позволяет быстро начать разработку и предоставляет все необходимые инструменты для создания функциональных ботов. Несмотря на некоторые недостатки, такие как потенциальные проблемы с производительностью и зависимость от внешних библиотек, преимущества pyTelegramBotAPI значительно перевешивают эти минусы, делая её предпочтительным выбором для большинства проектов.

Для студентов и начинающих разработчиков pyTelegramBotAPI может стать отличным стартом в мире разработки Telegram-ботов, предоставляя удобные инструменты и обширную документацию для быстрого освоения.

\subsubsection{schedule}
Библиотека schedule — это простая и удобная библиотека для планирования и выполнения периодических задач на языке программирования Python. Она широко используется для автоматизации рутинных задач и позволяет легко настроить выполнение функций в определенное время или через определенные интервалы. Рассмотрим основные аспекты этой библиотеки, её преимущества и недостатки.

Основные особенности schedule
Простота использования:

Библиотека обладает интуитивно понятным синтаксисом, что делает её доступной для начинающих разработчиков.
Задачи можно планировать с использованием простых и понятных команд, таких как every, day, hour, minute, second.
Гибкость в планировании задач:

Позволяет задавать сложные графики выполнения задач, такие как выполнение задач в определенные дни недели или месяца, а также в определенное время суток.
Поддерживает различные интервалы выполнения задач, от секунд до недель.
Легкая интеграция:

schedule легко интегрируется в существующие проекты на Python и может работать совместно с другими библиотеками и инструментами.
Минимальные зависимости:

Библиотека не требует установки большого количества дополнительных зависимостей, что упрощает её установку и использование.
Простота и удобство:

Библиотека очень проста в использовании и настройке. Разработчики могут быстро освоить её синтаксис и начать планировать задачи.
Легковесность:

schedule занимает мало места и не требует значительных ресурсов для работы, что делает её подходящей для небольших и средних проектов.
Минимальные зависимости:

Библиотека не требует сложной установки и настройки дополнительных пакетов, что упрощает её интеграцию в проекты.
Гибкость:

Поддерживает широкий спектр сценариев планирования, что позволяет создавать сложные графики выполнения задач.
Недостатки schedule
Ограничения по масштабируемости:

Библиотека не предназначена для работы с большими нагрузками и может оказаться неэффективной для задач, требующих высокой производительности и параллельного выполнения.

Однопоточная работа:

schedule работает в однопоточном режиме, что может стать проблемой для приложений, требующих параллельной обработки задач.

Отсутствие встроенного управления ошибками:

Библиотека не предоставляет встроенных средств для обработки ошибок и восстановления после сбоев, что требует дополнительной обработки исключений в коде.

Заключение:
Библиотека schedule является отличным инструментом для простого и удобного планирования задач в Python. Её простота и интуитивно понятный интерфейс делают её доступной для разработчиков любого уровня, а гибкость в настройке задач позволяет использовать её в широком спектре проектов. Однако для приложений, требующих высокой производительности и параллельной обработки задач, могут потребоваться более сложные решения.

\subsubsection{JSON}

Выбор JSON для хранения данных

JSON (JavaScript Object Notation) - это легкий формат обмена данными, который легко читается и записывается людьми и машинами. JSON широко используется в веб-разработке и является популярным выбором для хранения данных благодаря своей простоте и универсальности.

Преимущества использования JSON для хранения данных:
Читаемость и простота:

Читаемость: JSON легко читается и понимается как людьми, так и машинами. Это делает его удобным для разработчиков и администраторов.
Простота: JSON использует минимальный синтаксис, что упрощает создание и парсинг данных. В отличие от XML, JSON не требует закрывающих тегов и лишних элементов.
Универсальность:

Совместимость: JSON является языконезависимым форматом и может использоваться в большинстве языков программирования, таких как JavaScript, Python, Java, C\#, PHP и другие.
Широкая поддержка: JSON поддерживается многими библиотеками и фреймворками, что облегчает его интеграцию в различные проекты.
Легковесность:

JSON занимает меньше места по сравнению с другими форматами хранения данных, такими как XML. Это позволяет уменьшить объем передаваемых данных и повысить производительность приложений.
Гибкость:

JSON позволяет легко иерархически структурировать данные, что удобно для хранения сложных объектов и массивов.
Возможность вложенности позволяет эффективно организовывать данные.
Недостатки использования JSON для хранения данных:
Отсутствие схемы:

В отличие от XML, JSON не поддерживает встроенные схемы для валидации структуры данных. Это может привести к ошибкам, если данные не соответствуют ожидаемому формату.
Безопасность:

JSON-данные могут быть уязвимы для атак, таких как JSON-инъекции, если не применяются правильные меры безопасности. Это особенно важно при работе с пользовательским вводом.
Ограничения в типах данных:

JSON поддерживает ограниченное количество типов данных: строки, числа, объекты, массивы, логические значения и null. Более сложные типы данных, такие как даты, необходимо представлять в виде строк или других форматов.
Производительность:

Парсинг и сериализация JSON может быть медленнее по сравнению с бинарными форматами данных, такими как Protocol Buffers или MessagePack. Это может быть критично для высоконагруженных систем.
Заключение
JSON является отличным выбором для хранения данных благодаря своей простоте, читаемости и универсальности. Он идеально подходит для веб-разработки и обмена данными между клиентом и сервером. Однако необходимо учитывать его ограничения, такие как отсутствие встроенной схемы и возможные проблемы с безопасностью. Для приложений с высокими требованиями к производительности и безопасности могут потребоваться дополнительные меры и альтернативные форматы данных.

Таким образом, выбор JSON для хранения данных обоснован в большинстве случаев, особенно когда важны простота и совместимость.
\subsubsection{pathlib}

Библиотека pathlib является частью стандартной библиотеки Python и предоставляет объектно-ориентированный интерфейс для работы с файловыми системами. Она была введена в Python 3.4 как более современная и удобная альтернатива модулям os и os.path для работы с путями файловой системы.

Основные особенности библиотеки pathlib
Объектно-ориентированный интерфейс:

pathlib предоставляет классы для работы с путями, такие как Path и PurePath, что делает код более читабельным и удобным.
Кроссплатформенность:

Библиотека автоматически адаптируется к операционной системе, с которой работает, обеспечивая корректную обработку путей на разных платформах (Windows, macOS, Linux).
Удобство работы с путями:

Простое объединение, разделение и изменение путей с помощью перегруженных операторов и методов класса Path.
Расширенные возможности работы с файлами и директориями:

Поддержка создания, удаления, перемещения и копирования файлов и директорий.
Возможность получения информации о файлах и директориях, такой как размер, время последнего изменения и права доступа.
Работа с содержимым файлов:
Удобные методы для чтения и записи текстовых и бинарных файлов.

Преимущества библиотеки pathlib

Простота и удобство:
Объектно-ориентированный интерфейс делает работу с путями интуитивно понятной и удобной, улучшая читаемость и поддержку кода.
Кроссплатформенность:

pathlib автоматически обрабатывает различия между файловыми системами разных операционных систем, что упрощает разработку кроссплатформенных приложений.
Широкие возможности:

Библиотека предоставляет богатый набор методов для работы с путями, файлами и директориями, что позволяет выполнять большинство задач, связанных с файловой системой, без необходимости использования дополнительных библиотек.
Современный подход:

Включение pathlib в стандартную библиотеку Python отражает современные тенденции в разработке, делая её предпочтительным выбором для новых проектов.
Недостатки библиотеки pathlib
Переход с os и os.path:

Для разработчиков, привыкших использовать модули os и os.path, переход на pathlib может потребовать некоторого времени и усилий для адаптации.
Производительность:

Заключение:

Библиотека pathlib представляет собой мощный и удобный инструмент для работы с файловыми системами в Python. Её объектно-ориентированный интерфейс, кроссплатформенная совместимость и богатый набор возможностей делают её предпочтительным выбором для большинства задач, связанных с обработкой путей, файлов и директорий.

\subsection{Архитектура программной системы}

На рисунке 3.2 в виде UML-диаграммы показана архитектура программной системы.

\begin{figure}
\center{\includegraphics[width=1\linewidth]{un5}}
\caption{Архитектура программной системы}
\label{un5:image}
\end{figure}

Программно-информационная система содержит следующие компоненты:

1.	User Этот модуль представляет пользователя системы. Он хранит информацию о пользователе, такую как идентификатор, имя и местоположение.

2.	WeatherData Этот модуль представляет данные о погоде. Он хранит информацию о текущих погодных условиях для определенного местоположения.

3.	File Этот модуль представляет загруженные файлы. Он хранит информацию о файле, такую как идентификатор, имя и данные файла.

4.	UserPreferences Этот модуль хранит предпочтения пользователя, такие как предпочитаемый язык и настройки уведомлений.

5.	UserActivity Этот модуль хранит информацию о действиях пользователя в системе.

\subsection{Компоненты контроллера}
На рисунке 3.3 представлена UML-диаграмма классов-контроллеров.
\begin{figure}
	\center{\includegraphics[width=1\linewidth]{un6}}
	\caption{Архитектура программной системы}
	\label{un6:image}
\end{figure}
1. UserController – Класс отвечает за управление пользователями системы. Он обрабатывает запросы, связанные с регистрацией, аутентификацией и управлением пользовательскими данными.

2. WeatherController – Класс отвечает за получение и обработку данных о погоде. Он обрабатывает запросы, связанные с текущими погодными условиями для определенного местоположения. 

3. FileController - Класс отвечает за управление файлами, загруженными пользователями. Он обрабатывает запросы, связанные с загрузкой, скачиванием и удалением файлов.

4. PreferencesController – Класс отвечает за управление предпочтениями пользователя. Он обрабатывает запросы, связанные с обновлением и получением настроек пользователя.

5. ActivityController – Класс отвечает за управление действиями пользователя. Он обрабатывает запросы, связанные с получением и отслеживанием активности пользователя в системе. 

\subsection{Компоненты представления}

На рисунке 3.4 представлена UML-диаграмма классов- компонентов представления
\begin{figure}
	\center{\includegraphics[width=1\linewidth]{un7}}
	\caption{Архитектура программной системы}
	\label{un7:image}
\end{figure}

1. MainView - Основной компонент интерфейса, который отображает общую структуру и основные разделы приложения.

2. UserView– Компонент интерфейса для управления пользователем, включая регистрацию, аутентификацию и обновление информации.

3. WeatherView – Компонент интерфейса для отображения данных о погоде. 

4.FileView – Компонент интерфейса для управления файлами, загруженными пользователем.

5. PreferencesView – Компонент интерфейса для управления предпочтениями пользователя.

6. ActivityView – Компонент интерфейса для отображения активности 
\subsection{Проектирование пользовательского интерфейса программной системы}
\subsubsection{Основные цели и задачи}
Цель проектирования пользовательского интерфейса — создать удобный, интуитивно понятный и эффективный интерфейс для пользователей, который позволяет легко взаимодействовать с системой и выполнять необходимые действия.
\subsubsection{Общий подход к проектированию интерфейса}
Подход к проектированию интерфейса включает:

•	Анализ пользовательских требований и задач.

•	Создание макетов интерфейса.

•	Обеспечение интуитивной навигации и взаимодействия.

•	Тестирование интерфейса с реальными пользователями.

•	Внесение улучшений на основе отзывов пользователей.
\subsubsection{Основные компоненты интерфейса}
Для данного проекта, интерфейс будет включать несколько ключевых компонентов, обеспечивающих взаимодействие пользователя с ботом через мессенджер:

1.	Главное меню:

	-	Команда /start для запуска бота.
	
	-	Команда /help для получения списка доступных команд.
	
2.	Команды для работы с погодой:

	-	Команда /set\_city для установки города для получения прогноза погоды.
	
	-	Команда /current\_weather для получения текущей погоды.
	
	-	Команда /change\_city для изменения города.
	
3.	Обработка медиа сообщений:
	-	Возможность отправки текстовых сообщений, фотографий, видео и документов для их сохранения с ключевыми словами.


