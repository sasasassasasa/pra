\newsection
\centertocsection{ВВЕДЕНИЕ}

Для упрощения или автоматизации общения с клиентами создают специальные программы - чат-боты. Чат-бот это программный консультант, основная задача которого – выполнять за вас однотипные задачи, такие как ответить клиентам по статусу их заказа, оказание технической поддержки, рекламная рассылка и другое. Со стороны пользователей это обычный чат, но на самом деле пользователь общается со специальной программой.

\emph{Цель настоящей работы} – создание чат-бота, позволяющего помочь пользователям в решении их однотипных задач в группе или же канале Telegram. В процессе нужно использовать современные технологии, которые обеспечивают качественную и масштабируемую работу приложения, а также удовлетворяют потребности пользователей. Для достижения поставленной цели необходимо решить 
 \emph{следующие задачи:}
\begin{itemize}
\item провести анализ предметной области;
\item -	зарегистрировать бота в Telegram с помощью @BotFather и получить токен для доступа к API;
\item -	осуществить проектирование программной системы; разработать архитектуру программной системы;
\item провести тестирование работы программно-информационной системы.
\end{itemize}

\emph{Структура и объем работы.} Отчет состоит из введения, резюме стартап-проекта, 4 разделов основной части, заключения, списка использо- ванных источников, 2 приложений. Текст выпускной квалификационной рабооты равен \formbytotal{page}{страниц}{е}{ам}{ам}.

\emph{Во введении} цель работы, поставлены задачи разработки, описана структура работы, приведено краткое содержание каждого из разделов.

\emph{В резюме стартап-проекта}содержится основная информация о стартап-проекте: название, цели и стратегия, уникальность продукта, результаты, риски и перспективы проекта.

\emph{В первом разделе} на стадии описания технической характеристики предметной области приводится сбор информации о целевой аудитории, ее интересах и потребностях.

\emph{Во втором разделе} на стадии технического задания приводятся требования к разрабатываемому продукту.

\emph{В третьем разделе} на стадии технического проектирования представ- лены проектные решения для системы.

\emph{В четвертом разделе} приводится список классов, их атрибутов и методов, использованных при разработке чат-бота, производится тестирование разработанной системы.

В заключении излагаются основные результаты работы, полученные в ходе разработки.

В приложении А представлен графический материал.
В приложении Б представлены фрагменты исходного кода. 
